\input{vorspann}
\begin{document}

The treatment of form factors in light front holographic QCD
(LFHQCD) leads to particularly simple expressions for the
latter!\cite{}. From a dressed conserved current in the
4-dimensional holographic space $AdS_5$ , that is a current
propagating in the gravitational field, one obtains the general
results for form factors of hadrons with   $\ta$ constituents:
\beq \lb{FFcons} F_\ta(t) = \frac{1}{\cN_\ta} B\left[\ta-1, 1-
\frac{t}{4 \la}\right] \enq The current, which is conserved in
$AdS_5$ that is with an AdS  mass $\mu=0$, has in LFHQCD the
quantum numbers $J^P= 1^-$, since $\mu =0$ corresponds to
$J=L=1$~\cite{}. These are the quantum numbers of  an  axial
current and not of a vector current. For Form fasctors  vector
currents  one must expects  poles at the $\rho-omega$ meson and
its radial excitations and not at the axial vector meson poles.
Therefore we have shifted the second argument of the Beta function
to $\half - \frac{t}{4 \la}$ in order to have the poles at the
right positions~\cite{} \beq \lb{FFvec} F_\ta(t)^v =
\frac{1}{\cN_\ta} B\left[\ta-1, \frac{1}{2} - \frac{t}{4
\la}\right] \enq The quantity  $\la = \ka^2$ is the only scale in
LFHQCD with massless quarks and can be determined in this frame
from hadron spectroscopy.  From there the value $0.523 \pm 0.024 $
GeV$^2$ has been obtained~\cite{}. With this general result and
without any additional parameter good agreement with the
experimental data for the proton Dirac FF has been
obtained~\cite{}.

\textcolor{red}{ An expression like \req{FFcons} or \req{FFvec},
where the FF is expressed in terms of the Euler Beta function has
also been derived By Ademollo and del
Giudice~\cite{Ademollo:1969wd} in 1969. They extended the
Veneziano model for hadron scattering amplitudes, which was based
on the peculiar properties of the Beta function, to current
induced reactions and arrived at an expression for the vector FF
(without normalization):
 the result (without normalization):
\beq \lb{ad} F(t)=B[1-\al(t),\ga-1] \enq
 where $ \al(t)$ is a linear Regge trajectory ( the rho trajectory). }

Since the expression  \req{FFcons} describes  in LFHQCD the FF of
an axial vector current, it is natural to  use this expression the
calculation of the axial form factor. In this case we have no free
parameter, since the value of $\sqrt{\la} = \ka$ is determined
from hadron spectroscopy. In Fig. \ref{XXX} we show the result of
the prediction of the axial FF according to \req{FFcons}.

The discussion should cover the following points:

Reasonable, but not perfect fit, especially to small axial radius
when z-data are taken. Higher Fock state admixture can in the
range  be compensated by small variation of $\lambda$ within the
uncertainties.

Inserting the ''observed'' masses together with the counting-rule
constraint gives no good description of the data, only if we relax
the counting rule, we obtain a reasonable fit, comparable to ours.

Comparison with a free dipole fit


\textcolor{red}{Possible appendix:}

The Euler Beta function $B[x,y]= \frac{\Ga[x]\,\Ga[y]}{\Ga[x+y]}$
has two remarkable properties:

\beq \lb{reg} \lim_{x \to \infty,\, y \, \rm{fixed}} B[x,y]=
\Ga[y]\, x^{-y} \enq \beq \lb{dual} B[x,y] = \sum_{n=0}^\infty
\frac{(-1)^n}{n!} \frac{\Ga[y]}{\Ga[y-n]}\, \frac{1}{y+n} \enq
Because of the symmetry in $x$ and $y$ an analogous formula holds
for an expansion of poles in $y$.


These two properties induced Veneziano~\cite{Veneziano:1968yb} in
1968 to propose the Beta function as a model for hadron-hadron
scattering amplitudes. He made the following ansatz for a
scattering amplitude depending on the two invariants $s$ and
$t$~\footnote{possible crossing symmetries are not taken into
account here though thy played an important role in the
establishment and the discussion of the Veneziano model}  : \beq
\lb{ven} \cT(s,t) = \be B[1-\al(s), 1-\al(t)] \; \mbox{ with } \;
\al(s) = \al_{0}+ \al'\,s ;\;\al(t) = \al_{0}+ \al'\,t \enq The
intercepts $\al_0$ and the slopes $\al'$ for the two trajectories
can be different, essential is only the linear dependence.

Because of the properties (\ref{reg}) and \req{dual} this model
fulfilled two important structural constraints which were
important guidelines for strong interaction theory in the pre-QCD
time. 1)The amplitude can be either expressed as the sum over the
singularities in the $s$ {\bf or} the $t$-channel. 2) It fulfills
the so called Dolen-Horn-Schmidt duality, that is the sum over the
singularities in obe channel leads to Regge behaviour with the
Regge trajectory of the other channel. These properties are not
destroyed if one takes the sum of Beta
functions~\cite{Veneziano:1968yb} \beq \lb{vengen} \cT(s,t) =
\sum_{n,m=0} \be_n  B[1+n-\al(s), 1+m-\al(t)] \; \enq

 A peculiar property of the Veneziano model is the occurrence of daughter
  trajectories, that is trajectories with an intercept $\al_{0d}$ which is by a positive
  integer smaller than $\al_0$, the intercept of the parent trajectory, but the slopes are equal.
This is in accordance with the generalized form \req{vengen}.

Though the analytical duality and the Regge behaviour were the
main {\it raison d'\^etre} for the Veneziano model, the increasing
importance of sum rules derived from current algebra~\cite{} and
the occurrence of fixed ``Regge'' poles in current induced
reacyions~\cite{}  induced the investigation of Veneziano-like
amplitudes~\cite{Ademollo:1969wd,Bender:1970ew,Landshoff:1970ce}
where one of the arguments was not a variable trajectory (say
$\al_0+ \al'\,s $ with $\al'\neq 0$, but a constant integer
value~\footnote{an integer value since the current algebra
relations lead to fix poles at integer values}.   These could be
viewed as scattering amplitudes involving leptons, which do not
lead to resonance poles. Indeed, Ademollo and Del Giudice
\cite{Ademollo:1969wd} derived from from lepton-hadron elastic
scattering for the electromagnetic form factor the result (without
normalization): \beq \lb{ad} F(t)=B[1-\al(t),\ga-1] \enq Though
the structure is exactly the same as that of our result, there is
a subtle difference: In our expression the argument $\half -
\frac{t}{4 \la^2}$ is not a Regge trajectory in the sense that it
describes the resonances with increasing angular momentum, but is
describes the radial excitation. This is necessary, since all the
intermediate states in the $t$ channel must have the same quantum
numbers $J^P= 1^-$. In our model the slope of the radial
excitations is the same as that of the orbital excitations and
therefore this expression coincides with the Regge trajectory. The
result \req{ad} is also valid, because in the Veneziano the
daughters correspond exactly the radial excitations which are
Kaluza-Klein towers in $AdS_5$.


The ambiguity of the Venziano amplitude expressed in \req{vengen}
also applies to the derived form factor \req{ad} which then has
the form: \beq \lb{ad} F(t)=\sum_{n,m=0} \be_m\,
B[1-\al(t)+n,\ga-1+m ] \enq The contributions with $m>0$
correspond the higher twist contributions of LFHQCD.



\newpage

The following figure is for our information and discussion:


\includegraphics[width=16cm]{FFmult}


This table might be given in a modified form or not, but is a good
basis for the discussion,

\begin{table}\bec
\begin{tabular}{c c|c| c c |c}
&P&$\sqrt{\la}=\ka$ & 1. pole & 2. pole & $\chi^2$\\
&&GeV&GeV&GeV&\\
LFHQCD& 0 &0.510&1.020&1.442&0.019\\
LFHQCD& 0.3 &0.545&1.090&1.541&0.020\\
dipole,free&-&-&1.230&1.640&0.25\\
dipole,constr&-&-&1.230&1.640&1.20\\
\end{tabular} \enc
\caption{Different approximations for the axial FF.
$\sqrt{\la}=\ka$ is result of a fit to the model independent
z-data, 1. and 2. pole denotes the first and second pole mass;
$\ch^2= \sum_i({\rm data}_i  - {\rm fit}_i)^2/{\rm data}_i$ .
First two lines: LFHQCD with no  higher twist admixture $(P=0)$
and with admixture of $30 \%$. last two lines: Fit with poles at
observed $a_1(1260)$ mass and the unconfirmed $a_1(1640)$ mass,
free: without constraints on asymptotic behaviour, constr. :
asymptotic behaviour $\sim 1/t^2$. }
\end{table}























\end{document}
